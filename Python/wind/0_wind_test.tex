\documentclass[cs4size, UTF8]{ctexart}%
\usepackage[T1]{fontenc}%
\usepackage[utf8]{inputenc}%
\usepackage{lmodern}%
\usepackage{textcomp}%
\usepackage{lastpage}%
\usepackage[a4paper, left=2.5cm, right=2.5cm, top=3cm, bottom=3cm]{geometry}%
\usepackage{xeCJK}%
\usepackage{fontspec}%
\usepackage{mathptmx}%
\usepackage{amsmath}%
%
\title{\heiti 桥梁风荷载计算}%
\author{WYZ}%
\date{\today}%
\CTEXsetup[format={\raggedright\bfseries\large}]{section}%
\pagestyle{plain}%
%
\begin{document}%
\normalsize%
\maketitle%
\section{地质基本情况}%
\label{sec:}%
桥梁抗风风险区域:R2

        桥位地表分类:C

        十年重现期风作用水平:W1=19.500 m/s

        百年重现期风作用水平:W2=26.600 m/s
        

%
\section{风速参数}%
\label{sec:}%
根据规范4.1可得,基本风速值为%
\[%
U_{10}=26.600 \,m/s%
\]%

%
根据规范4.2.1可得,地表相关参数为%
\[%
\alpha_0=0.22 \quad z_0=0.30%
\]%

%
根据规范4.2.4可得,桥梁设计基本风速为%
\[%
U_{s10}=k_cU_{10}= 0.785\times26.600=20.881 \,m/s%
\]

%
\section{主梁风荷载}%
\label{sec:}%
根据规范4.2.2可得,主梁基准高度为%
\[%
Z=100.00 \,m%
\]%

%
根据规范4.2.6可得,主梁构件基准高度处的设计基准风速为%
\[%
U_d=k_f\left(\frac{Z}{10}\right)^{\alpha_0}U_{s10}= 1.02\times\left(\frac{100.00}{10}\right)^{0.22}\times20.881= 35.347 \,m/s%
\]%

%
根据规范5.2.1可得,等效静阵风风速为%
\[%
U_g=G_VU_d= 1.46\times35.347=51.783 \,m/s%
\]%

%
根据规范5.3.1可得,横桥向风作用下主梁单位长度上的顺风向等效静阵风荷载为%
\[%
F_g=\frac{1}{2}\rho U_g^2C_HD= 0.5\times1.25\times51.783^2\times1.530\times2.000= 5128.373 \,N/m%
\]

%
\end{document}