\documentclass[cs4size, UTF8]{ctexart}%
\usepackage[T1]{fontenc}%
\usepackage[utf8]{inputenc}%
\usepackage{lmodern}%
\usepackage{textcomp}%
\usepackage{lastpage}%
\usepackage[a4paper, left=2.5cm, right=2.5cm, top=3cm, bottom=3cm]{geometry}%
\usepackage{xeCJK}%
\usepackage{fontspec}%
\usepackage{mathptmx}%
\usepackage{longtable}%
\usepackage{tabu}%
\usepackage{amsmath}%
\usepackage{graphicx}%
%
\title{\heiti 桥梁桩长计算}%
\author{}%
\date{}%
\CTEXsetup[format={\raggedright\bfseries\large}]{section}%
\pagestyle{plain}%
%
\begin{document}%
\normalsize%
\maketitle%
\section{地基基本情况}%
\label{sec:}%
\begin{longtabu}{p{4cm}XXXXX}%
\hline%
\textbf{name}&\textbf{height}&\textbf{depth}&\textbf{fa0}&\textbf{qik}&\textbf{frk}\\%
\hline%
黏土&{-}3.1&3.1&130&60&0\\%
全风化凝灰岩&{-}5.5&2.4&150&55&0\\%
全风化凝灰岩&{-}9.1&3.6&150&55&0\\%
强风化凝灰岩&{-}10.5&1.4&450&90&0\\%
中风化凝灰岩&{-}17.5&7.0&1600&180&3700\\%
\hline%
\end{longtabu}

%
\section{桩基及其他参数取值}%
\label{sec:}%
桩基直径:$d=1.20\,m$

        桩基周长:$u=3.77\,m$

        桩基截面积:$A_p=1.13\,m^2$

        桩基密度:$\gamma=26.0\,kN/m^3$

        容许承载力随深度的修正系数:$k_2=1.5$

        各土层加权平均重度:$\gamma_2=18.0\,kN/m^3$

        清底系数:$m_0=0.7$
        

%
\section{桩长计算}%
\label{sec:}%
根据规范5.3.3可得,摩擦桩单桩承载力为%
\[%
[R_a] =\frac{1}{2}u\sum_{i=1}^nq_{ik}l_i+A_pq_r%
\]%

%
根据规范5.3.4可得,端承桩单桩承载力为%
\[%
[R_a]= c_1A_pf_{rk} +u\sum_{i=1}^mc_{2i}h_if_{rki} +\frac{1}{2}\xi_su\sum_{i=1}^nl_iq_{ik}%
\]%

%
考虑桩身自重与置换土重,桩基承载力为%
\[%
R_a =[R_a]-G_p+G_s%
\]%

%
代入不同长度桩长,可得摩擦桩与端承桩承载力如下图所示%


\begin{figure}[htbp]%
\centering%
\includegraphics[width=1\textwidth]{C:/Users/18817/AppData/Local/Temp/pylatex/5b623fae-b228-4d77-acdb-47aa5494aa7e.pdf}%
\end{figure}

%

%
不同桩长具体承载力如下表所示%
\begin{longtable}{p{1.5cm}|ll|ll}%
\hline%
桩长&摩擦桩承载力&安全系数&端承桩承载力&安全系数\\%
\hline%
0.0&36&0.01&0&0.00\\%
0.5&98&0.02&52&0.01\\%
1.0&160&0.04&104&0.02\\%
1.5&222&0.05&156&0.04\\%
2.0&284&0.07&208&0.05\\%
2.5&346&0.08&260&0.06\\%
3.0&408&0.10&312&0.07\\%
3.5&490&0.12&360&0.09\\%
4.0&548&0.13&408&0.10\\%
4.5&606&0.14&455&0.11\\%
5.0&664&0.16&502&0.12\\%
5.5&722&0.17&550&0.13\\%
6.0&780&0.19&597&0.14\\%
6.5&838&0.20&644&0.15\\%
7.0&896&0.21&692&0.16\\%
7.5&954&0.23&739&0.18\\%
8.0&1012&0.24&786&0.19\\%
8.5&1070&0.25&834&0.20\\%
9.0&1128&0.27&881&0.21\\%
9.5&1450&0.35&955&0.23\\%
10.0&1541&0.37&1035&0.25\\%
10.5&1632&0.39&1115&0.27\\%
11.0&2718&0.65&2584&0.62\\%
11.5&2894&0.69&2789&0.66\\%
12.0&3070&0.73&2994&0.71\\%
12.5&3245&0.77&3198&0.76\\%
13.0&3421&0.81&3403&0.81\\%
13.5&3597&0.86&3608&0.86\\%
14.0&3773&0.90&3813&0.91\\%
14.5&3949&0.94&4017&0.96\\%
15.0&4124&0.98&4222&1.01\\%
15.5&4300&1.02&4427&1.05\\%
16.0&4476&1.07&4631&1.10\\%
16.5&4652&1.11&4836&1.15\\%
17.0&4828&1.15&5041&1.20\\%
17.5&5003&1.19&5245&1.25\\%
\hline%
\end{longtable}%
由上述分析可知,当桩长为17.5m时,端承桩承载力为5245kN,安全系数为1.25,桩基承载力可满足规范要求。

%
\end{document}